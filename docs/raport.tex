\documentclass[a4paper,12pt]{article}
\usepackage[T1]{fontenc}
\usepackage[utf8]{inputenc}
\usepackage{amsmath, amssymb}
\usepackage{graphicx}
\usepackage{hyperref}
\usepackage{geometry}
\geometry{margin=1in}

\title{Raport: Komputery kwantowe}
\author{Jerzy Szyjut \\
        193064 \\
        Hubert Malinowski \\
        193088}
\date{\today}

\begin{document}

\maketitle

\section{Wstęp}
W ramach $\mu Projektu$ z przedmiotu Zaawansowane architektury komputerów 
przygotowaliśmy implementację kilku algorytmów kwantowych przy pomocy biblioteki
Qiskit w języku Python. W projekcie wzięliśmy pod uwagę algorytmy:
\begin{itemize}
    \item Grovera
    \item Shora
    \item Teleportaacji
\end{itemize}
Dzięki funkcjonalności biblioteki Qiskit algortmy te można uruchomić
na prawdziwych komputerach kwantowych, które są dostępne w chmurze
IBM.

\section{Wstęp teoretyczny}
Komputery kwantowe to urządzenia obliczeniowe, które wykorzystują 
zjawiska mechaniki kwantowej do przetwarzania informacji. Podstawową 
jednostką informacji w komputerze kwantowym jest \textbf{kubity} 
(ang. \emph{qubit}), który w przeciwieństwie do klasycznego bitu może 
znajdować się nie tylko w stanie $0$ lub $1$, ale także w ich superpozycji, 
czyli w stanie będącym kombinacją obu tych wartości:
\[
|\psi\rangle = \alpha|0\rangle + \beta|1\rangle,
\]
gdzie $\alpha$ i $\beta$ są liczbami spełniającymi $|\alpha|^2 + |\beta|^2 = 1$.

\vspace{0.5cm}

Kubity charakteryzują się dwoma kluczowymi właściwościami:
\begin{itemize}
    \item \textbf{Superpozycja} - możliwość jednoczesnego przebywania w wielu 
    stanach.
    \item \textbf{Splątanie} - zjawisko, w którym stan jednego kubitu jest 
    nierozerwalnie związany ze stanem innego, nawet jeśli są one od siebie 
    oddalone.
\end{itemize}

Obliczenia w komputerach kwantowych realizowane są za pomocą 
\textbf{bramek kwantowych}, które odpowiadają operacjom na kubitach. 
Bramki te, takie jak bramka Hadamarda, bramka Pauli-X czy bramka CNOT, 
pozwalają na manipulowanie stanami kubitów i tworzenie złożonych algorytmów 
kwantowych.

Komputery kwantowe mają potencjał do rozwiązywania pewnych problemów znacznie 
szybciej niż komputery klasyczne, wykorzystując równoległość obliczeń 
wynikającą z superpozycji oraz korelacje wynikające ze splątania.

\section{Uruchomienie projektu}

Aby uruchomić projekt wystarczy podążać za instrukcjami zawartymi w pliku 
\texttt{README.md}. Kopię tych instrukcji zamieszczam poniżej:


\subsection{Wymagania wstępne}

Aby rozpocząć pracę, upewnij się, że masz zainstalowane następujące elementy:

\begin{itemize}
    \item Python 3.11 lub nowszy
    \item Qiskit
    \item Jupyter Notebook
\end{itemize}

\subsection{Instalacja}

\begin{enumerate}
    \item Utwórz środowisko wirtualne (opcjonalnie, ale zalecane):
    \begin{verbatim}
python -m venv venv
source venv/bin/activate  # W systemie Windows użyj `venv\Scripts\activate`
    \end{verbatim}

    \item Zainstaluj wymagane zależności:
    \begin{verbatim}
pip install -r requirements.txt
    \end{verbatim}

    \item Uruchom Jupyter Notebook:
    \begin{verbatim}
jupyter notebook
    \end{verbatim}

    \item Otwórz notatniki Jupyter znajdujące się w katalogu \texttt{notebooks/}.

    \item Jeśli chcesz uruchomić kod na sprzęcie IBM Quantum, skonfiguruj swoje
     konto IBM Quantum i wprowadź token API w odpowiednim miejscu w kodzie.
\end{enumerate}

\subsection{Użytkowanie}

\begin{enumerate}
    \item Uruchom notatniki Jupyter w katalogu \texttt{notebooks/}.
    \item Modyfikuj i eksperymentuj z kodem, aby pogłębić swoją wiedzę.
\end{enumerate}

\section{Dostęp do komputerów kwantowych IBM}
Aby uzyskać dostęp do komputerów kwantowych IBM, należy zarejestrować się na
stronie IBM Quantum i uzyskać token API. Następnie można skonfigurować
token w kodzie, aby uzyskać dostęp do zasobów kwantowych. W momencie w którym
piszę ten raport, dostępne są komputery kwantowe o % uzupełnić
różnej liczbie kubitów, o architekturach różnych. W planie darmowym można
przeprowadzać obliczenia na komputerach kwantowych przez 10 minut miesięcznie.

\section{Budowa biblioteki Qiskit}

Qiskit to modułowa, open-source'owa biblioteka kwantowa rozwijana przez IBM i społeczność, zaprojektowana do tworzenia, analizowania i wykonywania obliczeń kwantowych. Architektura Qiskit została zbudowana z myślą o skalowalności i wsparciu dla różnych backendów kwantowych i klasycznych.

\subsection{Quantum Circuit}
Komponent \texttt{QuantumCircuit} stanowi podstawowy interfejs programowania algorytmów kwantowych. Umożliwia definiowanie rejestrów (klasycznych i kwantowych), aplikowanie bramek, pomiarów, a także kontrolę przepływu programu. Obwody są obiektami wysokiego poziomu, które można analizować, optymalizować oraz kompilować przy użyciu narzędzi takich jak \texttt{transpile}.

\subsection{Pass Manager}
\texttt{PassManager} odpowiada za kompilację obwodów kwantowych, przeprowadzając je przez szereg transformacji zwanych \textit{passami}. Każdy pass może optymalizować, przekształcać lub analizować obwód. Passy są organizowane w etapy (\textit{stages}) takie jak unifikacja bramek, mapowanie na topologię sprzętową oraz redukcja głębokości obwodu. Użytkownik może tworzyć własne sekwencje passów lub korzystać z domyślnych przepływów.

\subsection{Primitives}
\texttt{Primitives} to uproszczony i nowoczesny interfejs programistyczny umożliwiający użytkownikom wykonywanie zadań kwantowych takich jak próbkowanie (\texttt{Sampler}), szacowanie oczekiwanej wartości (\texttt{Estimator}) czy pomiary stanu. Odseparowują one szczegóły backendu i umożliwiają programowanie na wyższym poziomie abstrakcji, z zachowaniem precyzyjnej kontroli nad parametrami eksperymentu.

\subsection{Inne komponenty}
\begin{itemize}
  \item \textbf{Qiskit Runtime}: Architektura wykonawcza zapewniająca elastyczną i wydajną obsługę zadań kwantowych, działającą lokalnie lub w chmurze.
  \item \textbf{Qiskit Terra}: Podstawowy pakiet odpowiedzialny za kompilację i reprezentację obwodów.
  \item \textbf{Qiskit Aer}: Symulator kwantowy umożliwiający emulację obliczeń kwantowych na klasycznym sprzęcie.
  \item \textbf{Qiskit Experiments}: Framework do prowadzenia eksperymentów kalibracyjnych i charakterystyki urządzeń kwantowych.
\end{itemize}

\end{document}
